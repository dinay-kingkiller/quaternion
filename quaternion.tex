\documentclass{amsart}

\newtheorem{theorem}{Theorem}[section]
\newtheorem{lemma}[theorem]{Lemma}

\theoremstyle{definition}
\newtheorem{definition}[theorem]{Definition}
\newtheorem{example}[theorem]{Example}
\newtheorem{xca}[theorem]{Exercise}

\theoremstyle{remark}
\newtheorem{remark}[theorem]{Remark}

\numberwithin{equation}{section}

\begin{document}

\title{Quaternion Rotation}

\begin{abstract}
I've been distracted by quaternions and quaternion rotation since I first started playing with quadcopters. Hopefully this paper will hide my proofs from clogging up an article about robots. How can you not enjoy them? They're everything all at once.
\end{abstract}

\maketitle


\section{Quaternion Ring}

\begin{definition}[Quaternion]
  Formally, a quaternion consists of four values $(q_0, q_1, q_2, q_3)$ along with two operations: Addition $+$ and multiplication $\otimes$.
  Addition over the quaternions is defined by
  \begin{equation}
    \begin{pmatrix}
      p_0 \\
      p_1 \\
      p_2 \\
      p_3
    \end{pmatrix} +
    \begin{pmatrix}
      q_0 \\
      q_1 \\
      q_2 \\
      q_3
    \end{pmatrix} =
    \begin{pmatrix}
      p_0 + q_0 \\
      p_1 + q_1 \\
      p_2 + q_2 \\
      p_3 + q_3
    \end{pmatrix}
  \end{equation}
  Multiplication is more complex
  \begin{equation}
    \begin{pmatrix}
      p_0 \\
      p_1 \\
      p_2 \\
      p_3
    \end{pmatrix} \otimes 
    \begin{pmatrix}
      q_0 \\
      q_1 \\
      q_2 \\
      q_3
    \end{pmatrix} =
    \begin{pmatrix}
      p_0q_0 - p_1q_1 - p_2q_2 - p_3q_3 \\
      p_0q_1 + p_1q_0 + p_2q_3 - p_3q_2 \\
      p_0q_2 - p_1q_3 + p_2q_0 + p_3q_1 \\
      p_0q_3 + p_1q_2 - p_2q_1 + p_3q_0
    \end{pmatrix}
  \end{equation}
\end{definition}
\begin{theorem}
  Quaternions form a commutative group with addition.
\end{theorem}
\begin{proof}
  Quaternions exhibit associativity, since their members have associativity
  \begin{equation}
    \begin{pmatrix}
      p_0 + (q_0 + r_0) \\
      p_1 + (q_1 + r_1) \\
      p_2 + (q_2 + r_2) \\
      p_3 + (q_3 + r_3)
    \end{pmatrix} =
    \begin{pmatrix}
      (p_0 + q_0) + r_0 \\
      (p_1 + q_1) + r_1 \\
      (p_2 + q_2) + r_2 \\
      (p_3 + q_3) + r_3
    \end{pmatrix}
  \end{equation}
  They exhibit commutativity, since their members are commutative
  \begin{equation}
    \begin{pmatrix}
      p_0 + q_0 \\
      p_1 + q_1 \\
      p_2 + q_2 \\
      p_3 + q_3
    \end{pmatrix} =
    \begin{pmatrix}
      q_0 + q_0 \\
      q_1 + q_1 \\
      q_2 + q_2 \\
      q_3 + p_3
    \end{pmatrix} = (p + q) + r
  \end{equation}
  The additive identity is the \emph{zero quaternion} $0=(0, 0, 0, 0)$
  \begin{equation}
    \begin{pmatrix}
      q_0 + 0 \\
      q_1 + 0 \\
      q_2 + 0 \\
      q_3 + 0
    \end{pmatrix} =
    \begin{pmatrix}
      q_0 \\
      q_1 \\
      q_2 \\
      q_3
    \end{pmatrix}
  \end{equation}
  The additive inverse of quaternion $q$ is $-q=(-q0, -q1, -q2, -q3)$; subtraction is defined by the inverse $p-q = p+ (-q)$
  \begin{equation}
    \begin{pmatrix}
      q_0 + -q_0 \\
      q_1 + -q_1 \\
      q_2 + -q_2 \\
      q_3 + -q_3
    \end{pmatrix} =
    \begin{pmatrix}
      0 \\
      0 \\
      0 \\
      0
    \end{pmatrix}
  \end{equation}
\end{proof}

\begin{theorem}
  Quaternions form a group with the quaternion product.
\end{theorem}
\begin{proof}
  The quaternion product is associative. The proof is left to the reader. The left and right \emph{identity quaternion} is $(1, 0, 0, 0)$:
  \begin{equation}
    \begin{pmatrix}
      1q_0 - 0q_1 - 0q_2 - 0q_3 \\
      1q_1 + 0q_0 + 0q_3 - 0q_2 \\
      1q_2 - 0q_3 + 0q_0 + 0q_1 \\
      1q_3 + 0q_2 - 0q_1 + 0q_0
    \end{pmatrix} =
    \begin{pmatrix}
      q_0 \\
      q_1 \\
      q_2 \\
      q_3
    \end{pmatrix} =  
    \begin{pmatrix}
      1q_0 - 0q_1 - 0q_2 - 0q_3 \\
      0q_0 + 1q_1 + 0q_2 - 0q_3 \\
      0q_0 - 0q_1 + 1q_2 + 0q_3 \\
      0q_0 + 0q_1 - 0q_2 + 0q_3
    \end{pmatrix}
  \end{equation}
\end{proof}


Please note, quaternion multiplication is not commutative (For most quaternions: $p\otimes q\neq q \otimes p$). Some other algebraic properties:
Every scalar can be expressed as a quaternion $(s, 0, 0, 0)$ and every (3-)vector can be a quaternion $(0, v1, v2, v3)$. The 0-quaternion $(0, 0, 0, 0)$ is the additive identity and the 1-quaternion the multiplicative identity $(1, 0, 0, 0)$. The conjugate of a quaternion has inverted signs on the vector components: $q^*=(q0, q1, q2, q3)^*=(q0, -q1, -q2, -q3)$. The norm of a quaternion is the square root of the components, or the square of the quaternion multiplied by its conjugate (if we treat a purely scalar quaternion as a scalar).

\begin{equation}
  \|q\| = \sqrt{q\otimes q^*} = \sqrt{q_0^2+q_1^2+q_2^2+q_3^2}
\end{equation}
Every quaternion has an inverse $q^{-1}$ such that $q\otimes q^{-1}=1=q^{-1}\otimes q$. The inverse is $q^{-1}=q^*/\|q\|$. A quaternion with norm equal to one is called a unit quaternion, or versor. The quaternions covered outside this appendix can be considered unit quaternions. The $\otimes$ will be usually omitted too. We've seen that quaternion algebra can be applied to scalars and vectors. Pure vector multiplication notation will remain $\mathbf{u}\cdot\mathbf{v}$ and $\mathbf{u}\times\mathbf{v}$ for clarification, despite these also being quaternion operations.


% A division ring has:
% Addition is associative
% addition identity
% Addition inverse
% Addition commutivity
% Multiplication that is associative
% Multiplication is associative over addition
% Has an identity element
% Every nonzero element has a multiplicative inverse.

% \section{Quaternion Space}

% Vector space:
% Associativity of vector addition
% Commutativity of vector addition
% Vector addition identity
% Vector inverse identity
% Scaling Associativity a(bv) = (ab)v
% Scaling identity
% Distributivity of scalar over addition
% Distributivity of addition over scaling

\section{The Versor Subgroup}
\begin{definition}[Versors]
\end{definition}


% Closure
% Associativity
% Identity
% Inverse


\section{The Orthogonal Group}
A unit quaternion can be used to describe a rotation between bases. A vector described in unit basis $A$ would be notated $^A\mathbf{v}$. The quaternion $^Aq^B$ transforms $^A\mathbf{v}$ to $^B\mathbf{v}$ by the following formula:
\begin{equation}
  {^B\mathbf{v}} = \left({^Aq^B}\right)\left( {^A\mathbf{v}}\right)\left( {^Aq^{B*}}\right)
\end{equation}
By pre- and post-multiplying quaternion conjugates to invert the rotation $\left({^Aq^{B*}}\right)\left({^B\mathbf{v}}\right)\left({^Aq^B}\right)={^A\mathbf{v}}$, we see that the inverse of the rotation is just its conjugate. Since we are only talking about two reference frames in this paper: the inertial and the robot; rotating between the two can be described with a quaternion and its conjugate. Rotating from the inertial to the robot frame is chosen to be ${^Nq^R}=q$ and the robot to the inertial frame ${^Rq^N}=q^*$.

By Euler's rotation theorem, every rotation in 3-space can be described by a rotation axis and angle. For the transformation above, that quaternion is
\begin{equation}
  q = \exp\left(\frac{\theta}{2}\hat{\mathbf{u}}\right) = \cos\left(\frac{\theta}{2}\right) + \mathbf{u} \sin\left( \frac{\theta}{2}\right)
\end{equation}
where $\hat{\mathbf{u}}$ is the axis of rotation (scaled so that $\|q\|=1$) and $\theta$ is the rotated angle. Every rotation can be composed of small rotations when differentiated w.r.t. time gives:
\begin{equation}
  \dot{q} = \frac{1}{2} \dot{\theta} \ \exp\left(\frac{\theta}{2} \hat{\mathbf{u}} \right)
\end{equation}

\section*{Quaternion Kinematics}
Integrating the quaternion into

The quaternion is related to angular velocity by the following formula
\begin{equation}
  \dot{q} = \frac{1}{2} q \otimes \boldsymbol\omega
\end{equation}
%
\begin{definition}
  A quaternion rotates a 3-dimensional vector $\mathbf{u} \mapsto \mathbf{v}$
  \begin{equation}
    \mathbf{v} = \mathrm{q}\mathbf{u}\mathrm{q}^*
  \end{equation}


Quaternion rotation from angular velocity
\begin{equation}
  \dot{q} = \frac{1}{2} q\boldsymbol\omega
\end{equation}

Linear interpolation
\begin{equation}
  q_{k+1} = \frac{1}{2}q_k\boldsymbol\omega\Delta t + q_k 
\end{equation}

Spherical interpolation?
\begin{equation}
  q_{k+1} = \exp\left(\frac{1}{2} \boldsymbol\omega\Delta t\right)q_k
\end{equation}

\end{definition}



\end{document}

%-----------------------------------------------------------------------
% End of quaternion.tex
%-----------------------------------------------------------------------
