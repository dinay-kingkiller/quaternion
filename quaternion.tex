\documentclass{amsart}

\newtheorem{theorem}{Theorem}[section]
\newtheorem{lemma}[theorem]{Lemma}

\theoremstyle{definition}
\newtheorem{definition}[theorem]{Definition}
\newtheorem{example}[theorem]{Example}
\newtheorem{xca}[theorem]{Exercise}

\theoremstyle{remark}
\newtheorem{remark}[theorem]{Remark}

\numberwithin{equation}{section}

\begin{document}

\title{Quaternion Rotation}

\begin{abstract}
  Quaternions are almost magical with their ability to look like other concepts. They can be described by group/ring theory, or in terms of hyper-complex numbers, or in terms of vector spaces, and they can exist as all three of these at once. There are many resources on the algebra and analysis of quaternions, but often their definitions disagree, making it difficult to compare results. This paper hopes to formally define one type of quaternion operations and use those to derive final results. We will brush over proofs of generic quaternion algebra, and push through to applications ready for implementation in code.
\end{abstract}

\maketitle

\section{The Quaternion Ring}
% A division ring has:
% Addition is associative
% addition identity
% Addition inverse
% Addition commutivity
% Multiplication that is associative
% Multiplication is associative over addition
% Has an identity element
% Every nonzero element has a multiplicative inverse.


% setValue(m_floats[3] * q.x() + m_floats[0] * q.m_floats[3] + m_floats[1] * q.z() - m_floats[2] * q.y(),
% m_floats[3] * q.y() + m_floats[1] * q.m_floats[3] + m_floats[2] * q.x() - m_floats[0] * q.z(),
% _floats[3] * q.z() + m_floats[2] * q.m_floats[3] + m_floats[0] * q.y() - m_floats[1] * q.x(),
% m_floats[3] * q.m_floats[3] - m_floats[0] * q.x() - m_floats[1] * q.y() - m_floats[2] * q.z());

\begin{definition}[Quaternion]
  The Quaternion ring $\mathbb{H}$ consist of tuples of four real values $(x, y, z, w)\in\mathbb{R}^4$ along with some operations: Addition is defined easily enough
  \begin{equation}
    \begin{pmatrix}
      x_1 \\
      y_1 \\
      z_1 \\
      w_1
    \end{pmatrix} \oplus
    \begin{pmatrix}
      x_2 \\
      y_2 \\
      z_2 \\
      w_2
    \end{pmatrix} =
    \begin{pmatrix}
      x_1 + x_2 \\
      y_1 + y_2 \\
      z_1 + z_2 \\
      w_1 + w_2
    \end{pmatrix}
  \end{equation}
  Multiplication is more complex
  \begin{equation} \label{eqn:multiplication}
    \begin{pmatrix}
      p_0 \\
      p_1 \\
      p_2 \\
      p_3
    \end{pmatrix} \otimes 
    \begin{pmatrix}
      q_0 \\
      q_1 \\
      q_2 \\
      q_3
    \end{pmatrix} =
    \begin{pmatrix}
      p_0q_0 - p_1q_1 - p_2q_2 - p_3q_3 \\
      p_0q_1 + p_1q_0 + p_2q_3 - p_3q_2 \\
      p_0q_2 - p_1q_3 + p_2q_0 + p_3q_1 \\
      p_0q_3 + p_1q_2 - p_2q_1 + p_3q_0
    \end{pmatrix}
  \end{equation}
  Quaternions have a \emph{conjugation} operator $q\mapsto q^*$ like complex numbers
  \begin{equation}
    \begin{pmatrix}
      x \\
      y \\
      z \\
      w
    \end{pmatrix} ^ * =
    \begin{pmatrix}
      -x \\
      -y \\
      -z \\
      w
    \end{pmatrix}
  \end{equation}
  And they have a \emph{norm} that behaves like the length.
  \begin{equation}
    \|q\| = x^2 + y^2 + z^2 + x^2
  \end{equation}
  This formulation coincides with the ROS \texttt{tf2} package implementation. A less formal re-creation of some of these definitions can be found in the next section.
\end{definition}




\begin{theorem}
  Quaternions form a commutative group with addition, which means they 
  \begin{itemize}
    \item 
  \end{itemize}
  
\end{theorem}
\begin{proof}
  Quaternions exhibit associativity, since their members have associativity. They exhibit commutativity, since their members are commutative. The additive identity is the \emph{zero quaternion} $0=\left(0, 0, 0, 0\right)$. The additive inverse of quaternion $q$ is $-q=(-q0, -q1, -q2, -q3)$; subtraction is defined by the inverse $p-q = p+(-q)$.
\end{proof}

\begin{theorem}
  The quaternion product is associative. The proof is left to the reader.
\end{theorem}

\begin{theorem}
  The left and right identity of quaternion multiplication is the \emph{identity quaternion} $\mathbf{1}=(1,0,0,0)$
\end{theorem}
\begin{proof}
  \begin{equation}
    \begin{pmatrix}
      1q_0 - 0q_1 - 0q_2 - 0q_3 \\
      1q_1 + 0q_0 + 0q_3 - 0q_2 \\
      1q_2 - 0q_3 + 0q_0 + 0q_1 \\
      1q_3 + 0q_2 - 0q_1 + 0q_0
    \end{pmatrix} =
    \begin{pmatrix}
      q_0 \\
      q_1 \\
      q_2 \\
      q_3
    \end{pmatrix} =  
    \begin{pmatrix}
      1q_0 - 0q_1 - 0q_2 - 0q_3 \\
      0q_0 + 1q_1 + 0q_2 - 0q_3 \\
      0q_0 - 0q_1 + 1q_2 + 0q_3 \\
      0q_0 + 0q_1 - 0q_2 + 0q_3
    \end{pmatrix}
  \end{equation}
\end{proof}

\begin{definition}
  Quaternions form vector spaces: scaling quaternion $q$ by $s$ is defined in terms of quaternion multiplication $sq = (s,0,0,0)\otimes q$
\end{definition}

\begin{definition}
  All quaternions have a conjugate $q^*=(q_0,-q_1,-q_2,-q_3)$. 
\end{definition}

\begin{theorem}
  Conjugation is an involution $(q^*)^*=q$.
\end{theorem}

\begin{definition}
  The norm is the square root of the sum of the squares $\|q\|=\sqrt{q\otimes q^*}=\sqrt{q_0^2+q_1^2+q_2^2+q_3^2}$.
\end{definition}

\begin{theorem}
  $\|p\|\|q\|=\|pq\|$
\end{theorem}

\begin{theorem}
  Every quaternion, besides the zero quaternion, has an inverse $q^{-1}=\frac{q^*}{\|q\|}$
\end{theorem}

\begin{theorem}[Quaternion Commutor]
  Quaternion multiplication is not commutative.
\end{theorem}
\begin{proof}
  Let $p$ and $q$ be any two quaternions, then their commutor would be
  \begin{eqnarray}
    p\otimes q-q\otimes p &=&
    \begin{pmatrix}
      p_0q_0 - p_1q_1 - p_2q_2 - p_3q_3 \\
      p_0q_1 + p_1q_0 + p_2q_3 - p_3q_2 \\
      p_0q_2 - p_1q_3 + p_2q_0 + p_3q_1 \\
      p_0q_3 + p_1q_2 - p_2q_1 + p_3q_0
    \end{pmatrix} -
    \begin{pmatrix}
      p_0q_0 - p_1q_1 - p_2q_2 - p_3q_3 \\
      p_1q_0 + p_0q_1 + p_3q_2 - p_2q_3 \\
      p_2q_0 - p_3q_1 + p_0q_2 + p_1q_3 \\
      p_3q_0 + p_2q_1 - p_1q_2 + p_0q_3
    \end{pmatrix}\nonumber\\
    &=&
    \begin{pmatrix}
      0 \\
      2p_2q_3-2p_3q_2\\
      2p_3q_1-2p_1q_3\\
      2p_1q_2-2p_2q_1
    \end{pmatrix}
  \end{eqnarray}
\end{proof}

\subsection{Quaternion Notation}
Before this, the notation has been overly explicit to avoid confusion. But the value of quaternions come from their familiar algebraic properties. The motivation for defining quaternions had originally come from these extra special properties
\begin{definition}
  Quaternions behave like vectors, with scalars attached. Every vector $\mathbf{v} = \left(v_1, v_2, v_3\right)$ can be written as a quaternion $v = \left(0, v_1, v_2, v_3\right)$. A quaternion can be described, in turn, as a scalar and a vector $q = \left(q_0, \mathbf{q}\right)$, where $\mathbf{q} = \left(q_1, q_2, q_3\right)$. A quaternion with no vector part is a \emph{scalar quaternion}, able to be treated exactly like a scalar (commutivity, square roots, etc.). A quaternion with no scalar part (but with a vector part) is a \emph{pure quaternion} with additional operations like dot and cross products. Between them, scalars, vectors, and quaternions all have a way to define addition and multiplication. Those quaternion operations can also be re-written in terms of scalar and vector parts:
  \begin{equation}
    \begin{pmatrix}
      p_0 \\
      \mathbf{p}
    \end{pmatrix}
    +
    \begin{pmatrix}
      q_0 \\
      \mathbf{q}
    \end{pmatrix}
    =
    \begin{pmatrix}
      p_0+q_0 \\
      \mathbf{p} + \mathbf{q}
    \end{pmatrix}
  \end{equation}
  and
  \begin{equation}
    \begin{pmatrix}
      p_0 \\
      \mathbf{p}
    \end{pmatrix}
    \otimes
    \begin{pmatrix}
      q_0 \\
      \mathbf{q}
    \end{pmatrix}
    =
    \begin{pmatrix}
      p_0q_0 - \mathbf{p}\cdot \mathbf{q} \\
      p_0\mathbf{q} + q_0\mathbf{p} + \mathbf{p}\times \mathbf{q}
    \end{pmatrix}
  \end{equation}
  For pure quaternions, that final equation becomes
  $\mathbf{p}\otimes\mathbf{q} = \left( -\mathbf{p}\cdot \mathbf{q}, \mathbf{p} \times \mathbf{q}\right)$.
  The quaternion product $\otimes$ contains both the vector dot $\cdot$ and cross $\times$ products. For the rest of this paper, we will usually omit $\otimes$, but leave $\cdot$ and $\times$ for clarity.
\end{definition}

\begin{definition}
  Quaternions can act like complex numbers, by definining four orthogonal unit bases:
  \begin{equation}
    \mathbf{1} =
    \begin{pmatrix}
      1 \\
      0 \\
      0 \\
      0
    \end{pmatrix},
    \mathbf{i} =
    \begin{pmatrix}
      0 \\
      1 \\
      0 \\
      0
    \end{pmatrix}, 
    \mathbf{j} =
    \begin{pmatrix}
      0 \\
      0 \\
      1 \\
      0
    \end{pmatrix}, 
    \mathbf{k} =
    \begin{pmatrix}
      0 \\
      0 \\
      0 \\
      1
    \end{pmatrix},  
  \end{equation}
  Each quaternion can now be written out $q = q_1 \mathbf{1} + q_i \mathbf{i} + q_j \mathbf{j} + q_k \mathbf{k}$. The $\mathbf{1}$ is usually omitted, forcing in a definition for adding a scalar and a quaternion. This paper won't use this notation much, but it gives the best way to re-derive the multiplication defined back in equation \ref{multiplication}. From Hamilton's bridge, quaternion multiplication can be defined from the identities:
  \begin{equation}
    \mathbf{i}^2 = \mathbf{j}^2 = \mathbf{k}^2 = \mathbf{ijk} = -1
  \end{equation}
  Please note, some papers do not use this definition and therefor have a different definition of quaternions. The math can all work out the same at the end, but be wary swapping between papers could mean a missed sign or swapped commutativity.
\end{definition}




% \section{Quaternion Space}

% Vector space:
% Associativity of vector addition
% Commutativity of vector addition
% Vector addition identity
% Vector inverse identity
% Scaling Associativity a(bv) = (ab)v
% Scaling identity
% Distributivity of scalar over addition
% Distributivity of addition over scaling

\section{The Versor Subgroup}
% Closure
% Associativity
% Identity
% Inverse
  
\begin{definition}[Versors]
  Quaternions with norm $\|q\|=1$ form a subgroup of the quaternion product.
\end{definition}

\begin{theorem}
  The norm of the product of two quaternions is equal to the product of the norm of the two quaterions. So, if the two versors have norm equal to one, the norm of their product must be one. Which makes the resultant quaternion a versor.
\end{theorem}

\begin{theorem}
  The identity quaternion $(1, 0, 0, 0)$ belongs to the versor subgroup.
\end{theorem}

\begin{theorem}
  The inverse of every versor is a versor. As well, the inverse is equal to the conjugate.
\end{theorem}


\section{The Orthogonal Group}
% Closure
% Associative
% Identity
% Inverse
% SO(3) equivalence

\begin{definition}
  Quaternions can be used to describe rotations in space. A versor rotates between reference frames ${^\mathrm{A}q^\mathrm{B}}:\mathrm{A}\to\mathrm{B}$. Without describing $\mathbf{v}\in\mathbb{R}^3$ in either $\mathrm{A}$ or $\mathrm{B}$ basis:
  \begin{equation}
    \begin{pmatrix}
      \mathbf{v} \cdot \hat{\mathbf{b}}_x \\
      \mathbf{v} \cdot \hat{\mathbf{b}}_y \\
      \mathbf{v} \cdot \hat{\mathbf{b}}_z
    \end{pmatrix} = {^\mathrm{B}q^\mathrm{A}} \otimes 
    \begin{pmatrix}
      \mathbf{v} \cdot \hat{\mathbf{a}}_x \\
      \mathbf{v} \cdot \hat{\mathbf{a}}_y \\
      \mathbf{v} \cdot \hat{\mathbf{a}}_z
    \end{pmatrix} \otimes {^\mathrm{B}q^{*\mathrm{A}}}
  \end{equation}
\end{definition}

\begin{theorem}
  The \emph{unit quaternion} $\mathbf{1}$ is the rotation identity
  \begin{equation}
   \mathbf{1} \mathbf{v} \mathbf{1}^* = \mathbf{v}  
  \end{equation} 
\end{theorem}

\begin{theorem}
  By pre- and post-multiplying quaternion conjugates to invert the rotation
  \begin{equation}
    \left({^Aq^{B*}}\right)\left({^B\mathbf{v}}\right)\left({^Aq^B}\right)={^A\mathbf{v}}
  \end{equation}
  we see that the inverse of the rotation is just its conjugate. When there is just one rotation, the rotation will be denoted as $q$ and its inverse $q^*$.
\end{theorem}


\begin{theorem}[Quaternion to Matrix]
  Every rotation $\mathrm{A}\to\mathrm{B}$ can be described as both a quaternion and a rotation matrix
  \begin{equation}
    \left({^\mathrm{B}q^\mathrm{A}}\right) \otimes {^\mathrm{A}\mathbf{v}} \otimes \left({^\mathrm{A}q^\mathrm{B}}\right) = \left({^\mathrm{B}R^\mathrm{A}}\right){^A\mathrm{A}\mathbf{v}}
  \end{equation}
  that rotation matrix also collects the dot products of the rotation.
  
  \begin{tabular}{c|ccc}
    ${^BR^A}$ & $\hat{\mathbf{a}}_x$ & $\hat{\mathbf{a}}_y$ & $\hat{\mathbf{a}}_z$ \\
    \hline
    $\hat{\mathbf{b}}_x$ & $q_0^2+q_1^2-q_2^2-q_3^2$ & $-2q_0q_3+2q_1q_2$ & $2q_0q_2 + 2q_1q_3$ \\
    $\hat{\mathbf{b}}_y$ & $2q_0q_3+2q_1q_2$ & $q_0^2-q_1^2+q_2^2-q_3^2$ &  $-2q_0q_1 + 2q_2q_3$ \\
    $\hat{\mathbf{b}}_z$ & $-2q_0q_2+2q_1q_3$ & $2q_0q_1+2q_2q_3$ & $q_0^2-q_1^2-q_2^2+q_3^2$
  \end{tabular}\centering
\end{theorem}
\begin{proof}
  $R$ can be formed from algebra. The notation used implies that a pure quaternion is equal to its vector part.
  \begin{eqnarray}
    R \mathbf{v} &=&
    \begin{pmatrix}
      q_0 \\
      q_1 \\
      q_2 \\
      q_3
    \end{pmatrix} \otimes
    \begin{pmatrix}
      0 \\
      v_1 \\
      v_2 \\
      v_3 
    \end{pmatrix} \otimes
    \begin{pmatrix}
      q_0 \\
      -q_1 \\
      -q_2 \\
      -q_3
    \end{pmatrix} \nonumber \\
    R &=&
    \begin{pmatrix}
      q_0^2+q_1^2-q_2^2-q_3^2 & -2q_0q_3+2q_1q_2 & 2q_0q_2 + 2q_1q_3 \\
      2q_0q_3+2q_1q_2 & q_0^2-q_1^2+q_2^2-q_3^2 &  -2q_0q_1 + 2q_2q_3 \\
      -2q_0q_2+2q_1q_3 & 2q_0q_1+2q_2q_3 & q_0^2-q_1^2-q_2^2+q_3^2
    \end{pmatrix}
  \end{eqnarray} 
\end{proof}

\section{Smooth Transforms}
\begin{theorem}
  The quaternion is related to angular velocity by the following formula
  \begin{equation}
    \dot{q} = \frac{1}{2} q \otimes \boldsymbol\omega
  \end{equation}
\end{theorem}
\begin{proof}
  Start with the equation from 2-point theory, which says, for a vector $\mathbf{v}$, described in $\mathrm{A}$ (as ${^A\mathbf{v}}$) and in $\mathrm{B}$ (as ${^B\mathbf{v}}$):
  \begin{equation}
    \frac{d}{dt} {^\mathrm{B}\mathbf{v}} = \frac{d}{dt} {^\mathrm{A}\mathbf{v}} + {^\mathrm{B}\boldsymbol\omega^\mathrm{A}} \times {^\mathrm{A}\mathbf{v}}
  \end{equation}
  Strategically substituting our rotation quaternion
  \begin{equation}
    \frac{d}{dt} \left[\left({^\mathrm{B}q^\mathrm{A}}\right)\left({^A\mathbf{v}}\right)\left({^\mathrm{A}q^\mathrm{B}}\right) \right]  = \frac{d}{dt} {^\mathrm{A}\mathbf{v}} + {^\mathrm{B}\boldsymbol\omega^\mathrm{A}}\times {^\mathrm{A}\mathbf{v}}
  \end{equation}
  After this, we'll drop the $A$/$B$ bases for clarity. Choose a vector $\mathbf{v}$ in $\mathrm{A}$ where $\mathbf{v}$ is fixed, then:
  \begin{equation}
    \dot{q}\mathbf{v}q^*+q\mathbf{v}\dot{q}^* = \boldsymbol\omega\times \mathbf{v}
  \end{equation}
  Manipulating the left hand side with identities
  \begin{eqnarray}
    q^*\dot{q}\mathbf{v}q^*\dot{q}+q^*q\mathbf{v}\dot{q}^*\dot{q} &=& q^*\left(\boldsymbol\omega\times\mathbf{v}\right)\dot{q}\nonumber \\
    q^*\dot{q}\mathbf{v}q^*\dot{q}+\mathbf{v} &=& q^*\left(\boldsymbol\omega\times\mathbf{v}\right)\dot{q}\nonumber \\
  \end{eqnarray}
\end{proof}

Linear interpolation
\begin{equation}
  q_{k+1} = \frac{1}{2}q_k\boldsymbol\omega\Delta t + q_k 
\end{equation}

Spherical interpolation?
\begin{equation}
  q_{k+1} = \exp\left(\frac{1}{2} \boldsymbol\omega\Delta t\right)q_k
\end{equation}





\end{document}

%-----------------------------------------------------------------------
% End of quaternion.tex
%-----------------------------------------------------------------------
