\documentclass{amsart}

\newtheorem{theorem}{Theorem}[section]
\newtheorem{lemma}[theorem]{Lemma}

\theoremstyle{definition}
\newtheorem{definition}[theorem]{Definition}
\newtheorem{example}[theorem]{Example}
\newtheorem{xca}[theorem]{Exercise}

\theoremstyle{remark}
\newtheorem{remark}[theorem]{Remark}

\numberwithin{equation}{section}

\begin{document}

\title{Quaternion Rotation}

%    Remove any unused author tags.

\begin{abstract}
I've been distracted by quaternions and quaternion rotation since I first started playing with quadcopters. Hopefully this paper will hide my proofs from clogging up an article about robots. How can you not enjoy them? They're everything all at once.
\end{abstract}

\maketitle


\section{Quaternion Ring}

\begin{definition}[Addition]
  Addition over the quaternion ring
\end{definition}


% A division ring has:
% Addition is associative
% addition identity
% Addition inverse
% Addition commutivity
% Multiplication that is associative
% Multiplication is associative over addition
% Has an identity element
% Every nonzero element has a multiplicative inverse.

% \section{Quaternion Space}

% Vector space:
% Associativity of vector addition
% Commutativity of vector addition
% Vector addition identity
% Vector inverse identity
% Scaling Associativity a(bv) = (ab)v
% Scaling identity
% Distributivity of scalar over addition
% Distributivity of addition over scaling

\section{The Versor Subgroup}

% Closure
% Associativity
% Identity
% Inverse


\section{Rotation Transforms}

%



\end{document}

%-----------------------------------------------------------------------
% End of amsart-template.tex
%-----------------------------------------------------------------------
